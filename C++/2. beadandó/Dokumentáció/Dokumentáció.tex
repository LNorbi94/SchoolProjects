\documentclass[12pt]{report}
\usepackage[a4paper,top=15mm, bottom=25mm, left=23mm, right=23mm]{geometry}
\usepackage[english]{babel}
\usepackage[utf8]{inputenc}
\usepackage{mathptmx}
\usepackage{amssymb}
\usepackage{amsmath}
\usepackage{tabularx}
\usepackage{enumitem}
\usepackage{tikz}
\usepackage{stuki}

\renewcommand{\labelitemi}{$-$}
\renewcommand{\labelitemii}{}
\setlength{\parindent}{0pt}
\newcommand{\forceindent}{\leavevmode{\parindent=2.8em\indent}}

\usetikzlibrary{arrows,shapes,positioning,shadows,trees}

\tikzset{
  basic/.style  = {draw, text width=3cm, drop shadow, font=\sffamily, rectangle},
  root/.style   = {basic, rounded corners=2pt, thin, align=center,
                   fill=blue!30},
  level 2/.style = {basic, rounded corners=6pt, thin,align=center, fill=blue!20,
                   text width=15em}
}

\begin{document}
\begin{tabular}{lcr}
\textbf{Lestár Norbert} & 2. beadandó / 11. feladat  & 2015. április 24. \\
\textbf{Neptun kód:} A8UZ7T \\
lestar.norbert@gmail.com \\
6. csoport \\
\end{tabular}
\paragraph{Feladat} \hspace{0pt} \\
\textit{ Egy étteremben a pincérek által felvett rendeléseket egy szekvenciális input fájlban tartják nyilván az  ételek  neve,  azon  belül  a rendelések időpontja szerint  rendezett  formában. Feltehetjük, hogy a fájl nem üres. A tárolt adatok: a rendelt étel neve, a 
rendelés időpontja, rendelt  adagok száma, egy adag ára. Melyik étel hozta az étteremnek a legtöbb bevételt (összesített darab*egységár)? }
\paragraph{Specifikáció} \hspace{0pt} \\
A = ( t : Enor (Food), max : $\mathbb{S}$, e : E ) \newline
\forceindent Food = \textbf{rec}(id : $\mathbb{Z}$, name : $\mathbb{S}$, time : $\mathbb{S}$, count : $\mathbb{N}$, price : $\mathbb{N}$, income : $\mathbb{N}$) \newline
Ef = ( t = t$_{0}$ $\wedge$ $|$t$| >$ 0 ) \newline
Uf = ( (max, e) = MAX$_{e \in t_{0}} <$ e.income $>$ ) \newline
\paragraph{Absztrakt program} \hspace{2pt} \\ 

\begin{stukibox}[14cm]
\stm*{Maximum kiválasztás}
\stm[3]{t \rightarrow x sorainak felsoroloja \\
max, 0 \rightarrow max, < > \\
f(e) \rightarrow < e.income >
}
\end{stukibox}%
\hfill \newline
\begin{stukibox}[14cm]
\stm*{t.First() }
\stm*{e, max := t.Current(), e.income }
\stm*{ t.Next() }
\begin{WHILE}{4}{\stm*{$\neg$t.End()}}
\stm{e := t.Current()}
\begin{IF}{1}
{\stm{max < e.income} }
\stm{max := e.income}
\ELSE
\stm*{SKIP}
\end{IF}
\stm{t.Next()}
\end{WHILE}
\end{stukibox}%

\paragraph{Felsoroló} \hspace{0pt} \\
\begin{table}[h]
\centering
\begin{tabular}{|l|l|}
\hline
enor(Food) & First(), Next(), Current(), End() \\ \hline
\begin{tabular}[c]{@{}l@{}}line : String, stat : Státusz, file : InFile(String) \\ actual : Food,\\ over: $\mathbb{L}$\end{tabular} & \begin{tabular}[c]{@{}l@{}}First() $\rightarrow$ actual.id := 0, stat, line, file : read; Next()\\ Next() $\rightarrow$ a táblázat után\\ Current() $\rightarrow$ actual \\ End() $\rightarrow$ over\end{tabular} \\ \hline
\end{tabular}
\end{table}

\forceindent A \textit{Next()} műveletnek az alábbi feladatot kell megoldania: \hspace{0pt} \\

\forceindent Adott egy sorokra tördelt szöveges állomány ahonnan már kiolvastuk az első sort (ez
van a sor változóban, ha sikeres volt ez a korábbi olvasás) és
soron következő sort kell feldolgoznunk. Adott a korábbi ételek blokkjának sorszáma
(actual.id). Először is átolvassuk az ez előtti üres sorokat. Ha nem találunk ezek után egy
nem üres sort, akkor a vége változót igazra állítjuk (nincs több étel). Ha találunk, akkor a
vége változót hamisra állítjuk, az actual.id értékét eggyel növeljük, és az actual többi mezőjét az
itt kezdődő sorok alapján kitöltjük: beállítjuk az étel nevét, megszámoljuk a rendelt ételek mennyiségét, továbbá kiszámoljuk hogy mekkora bevételre tettünk szert általuk. Az első vizsgálathoz soronként külön meg kell számolni a remdelések számát, majd a következő rendelés lépés előtt összesíteni azt. Az olvasást a következő üres sorig, vagy a fájl végéig tesszük. \hspace{0pt} \\

A$^{Next}$ = (file: infile(String), line: String, stat Státusz, over: $\mathbb{L}$, actual: Food)\newline
Ef$^{Next}$ = (file=file$^{1} \wedge$ line=line$^{1} \wedge$ stat=stat$^{1} \wedge$ actual=actual$^{1}$)\newline
Uf$^{Next}$ = (stat$^{2}$, line$^{2}$, file$^{2}$ = $\mathbf{select}_{line \in (line^{1},file^{1})}$(stat = corrupted $\vee$ line $\neq$ üres) $\wedge$ \newline
\forceindent (over = (stat$^{2}$=correct)) $\wedge$ \newline
\forceindent ($\neg$over $\rightarrow$ \newline
\forceindent actual.id = actual$^{1}$.id + 1 \newline
\forceindent $\wedge$ actual.count, stat, line, file = $\sum\limits_{line \in (line^{2}, file^{2})}^{line \neq \ddot{u}res}$ actual.count \newline
\forceindent $\wedge$ actual.income = actual.price * actual.count
)\newline

\begin{stuki*}{ Next() }
\begin{WHILE}{1}{\stm*{line=üres $\wedge$ stat = correct}}
\stm{stat, line, file : read}
\end{WHILE}
\stm*{over := stat=corrupted }
\begin{IF}{14}{ \stm*{$\neg$ over } }
\stm{ actual.id := actual.id + 1 }
\stm{ actual.count := 0 }
\stm{ actual.income := 0 }
\begin{WHILE}{8}{\stm*{ line $\neq$ üres $\wedge$ stat = correct }}
\stm{ stat, word, line : read }
\stm{ actual.name := word }
\stm{ stat, word, line : read }
\stm{ actual.price := word }
\stm{ stat, word, line : read }
\stm[2]{ actual.count :=\\ actual.count + word }
\stm{ stat, line, file : read }
\end{WHILE}
\stm[2]{ actual.income 
:= \\ actual.price * actual.count }
\ELSE
\stm*{SKIP}
\end{IF}
\end{stuki*}%

\paragraph{Implementáció} \hspace{0pt} \\
\textit{Program váz} \\
A program több állományból áll: main.cpp, class.h, class.cpp.
\pagebreak

\begin{table}[h]
\begin{tabular}{|l|l|l|}
\hline
\textbf{main.cpp} & \textbf{class.h}                                                                                                                          & \textbf{class.cpp}                                                  \\ \hline
int main()        & \begin{tabular}[c]{@{}l@{}}struct Food\\ enum Status\\ class Enor\\ Enor ()\\ void First ()\\ Food Current () \\ bool End ()\end{tabular} & \begin{tabular}[c]{@{}l@{}}void Read ()\\ void Next ()\end{tabular} \\ \hline
\end{tabular}
\end{table}
Függvények kapcsolódási szerkezete: \\
main()-$>$ Enor()\\
\forceindent -$>$ First()\leavevmode{\parindent=1.2em\indent}-$>$ Read()\\
\forceindent -$>$ Next()\leavevmode{\parindent=1.21em\indent}-$>$ Read()\\
\forceindent -$>$ End()\\
\forceindent -$>$ Current() \newline
Felsoroló osztálya
\begin{table}[h]
\begin{tabularx}{\textwidth}{|X|l}
\hline
\begin{tabular}[c]{@{}l@{}} \textbf{class} Enor$\lbrace$\\
\forceindent\textbf{private:}\\ \forceindent\forceindent std::ifstream file;\\ \forceindent\forceindent std::stringstream line;\\ \forceindent\forceindent Status stat;\\ \forceindent\forceindent Food actual;\\ \forceindent\forceindent \textbf{bool} over;
\\ \forceindent\forceindent \textbf{void} Read();\\
\\ \forceindent\textbf{public:}\\ \forceindent\forceindent Enor(\textbf{const} std::string \&str);\\ \forceindent\forceindent \textbf{void} First() $\lbrace$ actual.id = 0; Read(); Next(); $\rbrace$\\ \forceindent\forceindent \textbf{void} Next();\\ \forceindent\forceindent Food Current() \textbf{const} $\lbrace$ \textbf{return} actual; $\rbrace$\\ \forceindent\forceindent \textbf{bool} End() \textbf{const} $\lbrace$ \textbf{return} over; $\rbrace$\\  $\rbrace$;\end{tabular} \\ \hline
\end{tabularx}
\end{table}

A szöveges állomány soronként olvasását a getline(file, line) utasítás végzi, amelyet beágyazunk a Read() metódusba úgy, hogy a linet stringstream típusú adatként adja vissza, illetve beállítsa az olvasás stat státuszát. \newline
A stringstream típusú line szavainak olvasása a >> operátorral történik. Üres egy sor, ha string üres. \newline
\pagebreak
\paragraph{Tesztelési terv} \hspace{0pt} \\
A megoldásban két helyen alkalmaztunk programozási tételt, a bekezdések és a sorok szintjén. \newline
\begin{itemize}[noitemsep]
\item Ételek szintjén maximum kiválasztás tesztesetei:
\begin{itemize}[noitemsep]
\item \textbf{intervallum hossza} szerint
\begin{enumerate}[noitemsep]
\item Üres állomány.
\item Csak üres sorokat tartalmazó állomány.
\item Egyetlen ételt tartalmazó állomány.
\item Több ételt tartalmazó állomány.
\end{enumerate}
\end{itemize}
\begin{itemize}[noitemsep]
\item \textbf{intervallum eleje és vége} szerint
\begin{enumerate}[noitemsep]
\item Első étel "a maximum".
\item Utolsó étel "a maximum".
\item Több étel van, és nem az első vagy utolsó a maximum.
\end{enumerate}
\end{itemize}
\end{itemize}
\begin{itemize}[noitemsep]
\item Egy ételhez tartozó sorok szintjén intervallum és egy összegzés egybefoglalt tesztesetei:
\begin{itemize}[noitemsep]
\item \textbf{intervallum hossza} szerint
\begin{enumerate}[noitemsep]
\item Egyetlen nem üres sor.
\item Több nem üres sor.
\end{enumerate}
\end{itemize}
\begin{itemize}[noitemsep]
\item \textbf{intervallum eleje és vége} szerint
\begin{enumerate}[noitemsep]
\item Több sor, ahol csak a legelső nap történt rendelés.
\item Több sor, ahol csak a legutolsó nap történt rendelés.
\end{enumerate}
\end{itemize}
\end{itemize}
\end{document}